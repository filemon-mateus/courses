\documentclass[10pt, letterpaper]{article}
\usepackage{preamble}

\begin{document}
\header{CS 6150}{Filemon Mateus}{Homework \#\ 7}{Computational Complexity}{\today}
\begin{enumerate}[label={\bfseries Q\arabic*.}]
  \item
    \begin{enumerate}
      \item
        To reduce \textsc{BoxDepth} to \textsc{Clique}, we take an arbitrary instance of
        \textsc{BoxDepth} and convert it into an instance of \textsc{Clique}. Specifically
        (1) we construct a graph with one vertex for each rectangle, and (2) for each pair
        of rectangles that intersect, we add an edge in the graph between the corresponding
        vertices. Assuming we are given $n$ rectangles in \textsc{BoxDepth}, then we note
        that (1) and (2) induce a polynomial transformation on the size of the original
        problem instance---$n$. This is because the construction of the vertices in (1) takes
        at most $O(n)$, and adding an edge between two vertices in (2) amounts to deciding
        whether or not their corresponding rectangles in \textsc{BoxDepth} intersect. This is a task
        that can be done computationally in $O(1)$ via intersecting the intervals projected
        on $x$-axis and $y$-axis. Since, we are given $n$ rectangles initially, then we are
        guaranteed exactly $n(n-1)$ decision problems in (2), each of which incurs a constant
        cost, yielding a total transformation cost of $O(n^2)$ for (2). Now, since (1) and
        (2) are polynomial in $n$, our transformation is guaranteed to be polynomial as well
        in the size of the original problem instance---$n$. \\

        We now show why an instance of \textsc{BoxDepth} and an instance of \textsc{Clique}
        on the graph induced by (1) and (2) are equivalent to one another via reduction. \\

        \begin{proof*}
          We include both directions and demonstrate that an instance of \textsc{BoxDepth}
          has a depth of $k$ \textit{iff} the corresponding vertices of their intersecting
          rectangles form a clique of size $k$ on the graph induced by (1) and (2).
          \begin{description}[topsep=5pt, itemsep=10pt]
            \item[$\Longrightarrow$]
              ({\itshape Necessary Condition}) --- follows trivially from the definition of (1)
              and (2). Specifically, if an instance of \textsc{BoxDepth} has a depth of $k$---meaning
              that $k$ rectangles intersect in some common point---then clearly the corresponding
              vertices in the constructed graph induced by (1) and (2) will have a clique of size $k$.
              This is because if a common point belongs to the pairwise intersection of any two
              rectangles, then any two corresponding vertices are connected by an edge, and thus
              the $k$ vertices (representing the $k$ intersecting rectangles) form a clique in
              the graph induced by (1) and (2).

            \item[$\Longleftarrow$]
              ({\itshape Sufficient Condition}) --- conversely, if the constructed graph induced by
              (1) and (2) forms a clique of size $k$, with $k > 1$, then we are guaranteed that the corresponding
              rectangles are in such way that every pair of them intersect (not necessarily that they
              result in a depth of $k$). But since every pair of them intersect then all of the
              projections of the rectangles on the $x$-axis and $y$-axis also intersect pairwise, and
              we can use \textbf{\textsc{Helly's Theorem}}\footnote{\textbf{\textsc{Helly's Theorem}}.
              {\itshape Let $K$ be a family of at least $d+1$ convex sets in $\mathbb{R}^d$, and assume
              $K$ is finite or that each member of $K$ is compact. Then if each $d+1$ members of $K$
              have a common point, there is a point common to all members of $K$}. 
              % Note \textbf{\textsc{Helly's Theorem}} only claims intersection of all projections
              % of the rectangles on the $x$ and $y$-axes but this is sufficient for intersection
              % on the plane as well because the rectangles are axis aligned.
              Definition outline weakly
              adapted from: \href{https://en.wikipedia.org/wiki/Helly's\_theorem}{\url{https://en.wikipedia.org/wiki/Helly's\_theorem}}.}
              in $\mathbb{R}$ to conclude that the rectangles have a \textit{common} intersection and
              hence have a depth of $k$. For special cases of cliques of size $k = 0$ and $k = 1$, the
              converse follows vacuously (when $k = 0$) and trivially (when $k = 1$) from the definition
              of \textsc{BoxDepth}.
          \end{description}
        \end{proof*}

      \item
        \textit{Algorithm Description and Time Analysis:} \par
        We consider all the projections of the rectangles on the $x$-axis and $y$-axis. There are at
        most $2n$ distinct points, which we can sort in $O(n \log n)$, so at most $2n+1$ intervals per
        axis. It suffices to consider open intervals. We select one point from at most $(2n+1)(2n+1) =
        O(n^2)$ regions (which we can index in a 2D array), and check in $O(1)$ per rectangle if the
        point is included ($O(n)$ per point), storing one counter per region. Finally, we iterate over
        the array and compute the maximum. This yields $O(n^3)$ overall.

      \item
        Since the reduction is to a \textsc{Clique}, part (a) does not show that \textsc{BoxDepth} is
        NP-complete. Had we shown that an NP-complete problem (such as \textsc{Clique}) reduces to a
        problem in P (such as \textsc{BoxDepth}), we could have drawn this conclusion.
    \end{enumerate}

  \item
    By reducing from \textsc{3SAT}, we demonstrate that this puzzle is NP-hard. Specifically, given
    an instance $\Phi$ of \textsc{3SAT} in $n$ clauses and $m$ variables, we proceed with a polynomial
    transformation of the board into a puzzle configuration as follows. For all indices $i$ and $j$,
    with $1 \leq i \leq n$ and $1 \leq j \leq m$, we fill each board entry $(i,j)$ with a ``snowman''
    whenever the boolean variable $x_j$ is part of the $i$-th clause of $\Phi$. By symmetry, we add
    a ``snowflake'' at position $(i,j)$ whenever the negation $\overline{x}_j$ is part of the $i$-th
    clause of $\Phi$. Finally, we default to empty spaces if neither $x_j$ nor its negation $\overline{x}_j$
    are part of the $i$-th clause of $\Phi$. \\

    \begin{claim}
      \label{claim:1}
      We claim this puzzle configuration is solvable if and only if $\Phi$ is satisfiable.
    \end{claim}
    \begin{proof*}
      We include both directions of of \autoref{claim:1} in the proof outline below.
      \begin{description}[topsep=5pt, itemsep=10pt]
        \item[$\Longrightarrow$]
          ({\itshape Necessary Condition}) --- assume the puzzle can be solved. Then we consider a
          solution construction where for each column index $j$ we make an assignment to $x_j$ that
          is \textsc{True} if column $j$ contains ``snowmen,'' \textsc{False} if column $j$ contains
          ``snowflakes,'' and arbitrary values of \textsc{True/False} if column $j$ is empty. That
          is, assign values to the variables so that the literal corresponding to the remaining
          pieces are all \textsc{True}. Each row still has at least one puzzle piece, so each
          clause of $\phi$ contains at least one \textsc{True} literal, so this construction makes
          $\phi=\textsc{True}$. Hence, we conclude $\phi$ is satisfiable.

        \item[$\Longleftarrow$]
          ({\itshape Sufficient Condition}) --- suppose $\phi$ is satisfiable; consider an arbitrary
          satisfying assignment. For each index entry $j$ remove ``snowmen'' from column $j$ according
          to the value assigned to $x_j$: (1) if $x_j=\textsc{True}$, remove all ``snowflakes'' from
          column $j$; (2) if $x_j$ remove all ``snowmen'' from column j. That is, remove the puzzle
          pieces that correspond to \textsc{False} literals. Now, because every variable appears in
          at least one clause, each column contains at least one \textsc{True} literal, and thus each
          row still contains at least one puzzle piece. We conclude that the puzzle is satisfiable.
      \end{description}
    \end{proof*}

    We only need linear time to perform this reduction, since we can allocate a column to a new variable
    as we go. Hence the reduction is polynomial-time. In conclusion, this decision problem on the solvability
    of the puzzle is NP-hard.
\end{enumerate}
\end{document}
