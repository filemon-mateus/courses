\documentclass[10pt, letterpaper]{article}
\usepackage{preamble}

\begin{document}
\header{CS $6150$}{Filemon Mateus}{Homework \#\ $5$}{Dynamic Programming}{\today}
\begin{enumerate}[label={\bfseries Q\arabic*.}]
  \item
    \begin{enumerate}
      \item
        {\itshape Algorithm Description} \\ \vspace{-4mm}

        We proceed with dynamic programming. Specifically, for any integer $0 \leq i \leq n$ and $0 \leq j
        \leq M$, we split the original problem into individual subproblem instances of the form $f[i][j]$,
        each representing a boolean value that is \textsc{True} \textit{iff} there exists a multiset of
        items $1, \ldots, i$ which size exactly $j$. With a abuse of set notation we define:
        \[
          f[i][j] = 1 \iff \exists\ S \subseteq \{1, \ldots, i\} \ni \sum_{i \in S} a_i = j
        \]
        At iteration $i$, we discuss the progress made by our dynamic program in one of the following ways:
        \begin{itemize}
          \item[(1)]
            Item $i$ occurs at least once in $S$. In this case, we select a multiset of items $1, \ldots, i$
            that size exactly $j-a_i$. Here $f[i][j-a_i]$ indicates whether such multiset exists.

          \item[(2)]
            Item $i$ does not occur in $S$. In this case, the solution maps to a multiset of the items $1,
            \ldots, i-1$ that size exactly $j$. The answer of whether such multiset exists is found in $f[i
            -1][j]$.
        \end{itemize}

        Either one of the above avenues yields a solution, so $f[i][j]$ is \textsc{True} if at least one of
        the decision possibilities above results in a solution. Furthermore, we can formalize these decision
        states as a dependency relation among the subproblems of $f$. Namely:
        \[
          f[i][j] = \begin{cases}f[i-1][j] & \text{if}\ j < a_i \\ \max\big\{f[i-1][j], f[i][j-a_i]\big\} &
            \text{otherwise} \end{cases}
        \]
        The base case follows from the definition of $f[i][j]$, i.e., $f[i][0] = 1$ for all $0 \leq i
        \leq n$ and $f[0][j] = 0$ for all $1 \leq j \leq M$. The answer we are after is $f[n][M]$. Since we
        do not know where it resides, we try all possible pairs $(i,j)$ and check if any of them evaluate to
        \textsc{True}---note this is achieved by taking the max across the intermediate answers to $f[n][M]$.
        \\

        Note this is a two-dimensional dynamic programming problem with table size $(n + 1, M + 1)$. Each table entry
        is a boolean value (\textsc{True/False}) capturing the inclusion (or exclusion) of item $i$ in the final
        multiset $S$. Each such decision incurs an $O(1)$ work (deciding among the two previous precomputed
        subproblems which one is \textsc{True}) so the total running time is in the order of $O(nM)$.

      \item
        {\itshape Algorithm Pseudocode} \vspace{-4mm}
        \begin{center}
          \begin{minipage}{\linewidth}
            \begin{algorithm}[H]
              \caption{\textsc{Knapsack}$(A, M)$}\label{alg:unlimited-knapsack}
              \begin{algorithmic}[1]
                \Require{A list of $n$ item sizes $A[1 \cdots n]$ together with an integer $M$---the size of the
                knapsack.}
                \Ensure{{\color{red!50!black}``\texttt{yes}''} \textit{iff} there exists a multiset $S$ of items
                of $A$ whose total size is equal to $M$; {\color{red!50!black}``\texttt{no}''} otherwise.}
                \State $f[i][0] \gets 1$ \textbf{for each} $i \in \{0 \cdots n\}$
                \State $f[0][j] \gets 0$ \textbf{for each} $j \in \{1 \cdots M\}$
                \For{$i \gets 1$ \textbf{to} $n$}
                  \For{$j \gets 1$ \textbf{to} $M$}
                    \If{$j < A[i]$}.
                      \State $f[i][j] \gets f[i-1][j]$
                    \Else
                      \State $f[i][j] \gets \max\big\{f[i-1][j], f[i][j-A[i]]\big\}$
                    \EndIf
                  \EndFor
                \EndFor
                \If{$f[n][M] = 1$}
                  \State \Return {\color{red!50!black}``\texttt{yes}''}
                \Else
                  \State \Return {\color{red!50!black}``\texttt{no}''}
                \EndIf 
              \end{algorithmic}
            \end{algorithm}
          \end{minipage}
        \end{center}
    \end{enumerate}

  \newpage
  \item
    \begin{enumerate}
      \item
        {\itshape Algorithm Description} \\ \vspace{-4mm}

        We proceed with dynamic programming. Specifically, for any integer $0 \leq i \leq n$ and $0 \leq
        j \leq M$, we split the original problem into individual subproblem instances of the form $f[i][j]$
        denoting the maximum attainable value obtained using a knapsack of capacity $j$ and the first $i$
        items $1, \ldots, i$. At iteration $i$, we face a decision problem; that is, either item size $a_i$
        is needed to achieve the optimal solution, or it isn't needed. Since we want to maximize $\sum_{a_i
        \in S} v_i$, we select the most profitable result among the two following decision states:

        \begin{itemize}
          \item[(1)]
            Fill the knapsack with item $i$. In this case, we must select a subset of items $1, \ldots, i-1$
            that have a combined size at most $j - a_i$. Assuming we do all the preceding $i-1$ choices correctly,
            then we will get $f[i-1][j-a_i]$ value out of items $1, \ldots, i-1$, so the cumulative total
            will be $f[i-1][j-a_i] + v_i$.

          \item[(2)]
            Don't fill the knapsack with item $i$, so we'll re-use the optimal solution for items $1,
            \ldots, i-1$ that have an aggregate size at most $j$. That answer rests in $f[i-1][j]$.
        \end{itemize}

        We can formalize $(1)$ and $(2)$ as the following dependency relation among the subproblems of $f$:
        \[
          f[i][j] =
            \begin{cases}
              f[i-1][j] & \text{if}\ j < a_i \\
              \max\big\{f[i-1][j], f[i-1][j-a_i] + v_i\big\} & \text{otherwise}
            \end{cases}
        \]
        where the second parameter of max is invoked when item size $a_i$ with value $v_i$ is part
        of the solution subset. Note the initial conditions for this problem are $f[i][0] = 0$ for
        all $0 \leq i \leq n$ and $f[0][j] = 0$ for all $0 \leq j \leq M$. The answer we seek is
        $f(n, M)$. Since we do not know which pair $(i,j)$ yields $f(n, M)$, we try all possible
        combinations of $i$'s and $j$'s and pick the maximal among them. The pseudocode for this is
        provided below and boils down to filling out a two dimensional table of size $(n+1, M+1)$.
        Each entry of the table incurs an $O(1)$ work (deciding among the two previous precomputed
        subproblems which one is larger) so the total running time is $O(nM)$.

      \item
        {\itshape Algorithm Pseudocode} \vspace{-4mm}
        \begin{center}
          \begin{minipage}{\linewidth}
            \begin{algorithm}[H]
              \caption{\textsc{Knapsack}$(A, V, M)$}\label{alg:limited-knapsack}
              \begin{algorithmic}[1]
                \Require{A list of $n$ item sizes $A[1 \cdots n]$, its corresponding values $V[1 \cdots n]$,
                and an integer $M$---the size of the knapsack.}
                \Ensure{The maximum attainable value with a knapsack of size $M$.}
                  \State $f[i][0] \gets 0$ \textbf{for each} $i \in \{0 \cdots n\}$
                  \State $f[0][j] \gets 0$ \textbf{for each} $j \in \{0 \cdots M\}$
                  \For{$i \gets 1$ \textbf{to} $n$}
                    \For{$j \gets 1$ \textbf{to} $M$}
                      \If{$j < A[i]$}
                        \State $f[i][j] \gets f[i-1][j]$
                      \Else
                        \State $f[i][j] \gets \max\big\{f[i-1][j], f[i-1][j-A[i]] + V[i]\big\}$
                      \EndIf
                    \EndFor
                  \EndFor
                  \State \Return $f[n][M]$
              \end{algorithmic}
            \end{algorithm}
          \end{minipage}
        \end{center}
    \end{enumerate}

  \newpage
  \item
    \begin{enumerate}
      \item
        {\itshape Algorithm Description} \\ \vspace{-4mm}
        
        The \textit{maximum sum common subsequence} (MSCS) problem has identical decisions and states
        to the \textit{longest common subsequence} (LCS) problem discussed in class. So, without prizes
        for guessing, we proceed with dynamic programming. Specifically, for integers $0 \leq i \leq n$
        and $0 \leq j \leq m$, we introduce individual subproblem instances of the form $f[i][j]$
        representing the MSCS attainable from inputs $A[1 \cdots i]$ and $B[1 \cdots j]$. When computing
        $f[i]$[j], we take into account one of the following decision states:
        \begin{itemize}
          \item[(1)]
            $A[i] = B[j]$. Then clearly the MSCS is obtained by maximizing the sum of the subsequences
            $A[1 \cdots i-1]$ and $B[1 \cdots j-1]$, then adding $A[i]$ (or $B[j]$) to both subsequences.
            Notice, however, that there is a small caveat here, namely: $A[i]$ could be negative!
            Since we want to maximize the sum we add $A[i]$ only if $A[i]$ is positive. This should
            satisfiably handle negative numbers in $A[1 \cdots i]$ and $B[1 \cdots j]$ and make our dynamic
            program robust against edge cases of this nature.

          \item[(2)]
            $A[i] \neq B[j]$. Then clearly the optimal subsequence yielding MSCS is located in the
            shorter version of $A[1 \cdots i]$ and $B[1 \cdots j]$. Since we do not know which one it
            is, we solve both $f[i-1][j]$ and $f[i][j-1]$ and pick the corresponding result that yields
            the maximal sum among them.
        \end{itemize}
        Formalizing the aforementioned decision states as a dependency relation across the subproblems of
        $f$, we see that:
        \[ 
          f[i][j] = 
          \begin{cases}
            f[i-1][j-1] + \max\big\{A[i], 0\big\} & \text{if}\ A[i] = B[j] \\
            \max\big\{f[i][j-1], f[i-1][j]\big\} & \text{if}\ A[i] \neq B[j]
          \end{cases}
        \]
        The base case follows from the definition of $f[i][j]$, i.e., $f[i][0] = 0$ for all $0 \leq i \leq
        n$ and $f[0][j] = 0$ for all $0 \leq j \leq m$. We are after $f[n][m]$, but since we do not know
        which pairs of indices $(i, j)$ yield $f[n][m]$, we try all permissible choices of $i$'s and $j$'s
        and pick the maximal one among them. \\

        Due to the optimal subproblem structure similarity to LCS, we see that MSCS is also a two
        dimensional dynamic program of table size $(n+1, m+1)$. Retrieving the value of $f[i][j]$ at any given
        iteration takes $O(1)$. There are $(n+1)(m+1)$ entries in the table, so we get a final, total running
        time of $O(nm)$ as required! The pseudocode for this is provided below.

      \item
        {\itshape Algorithm Pseudocode} \vspace{-4mm}
        \begin{center}
          \begin{minipage}{\linewidth}
            \begin{algorithm}[H]
              \caption{\textsc{Max-Sum-Common-Subsequence}$(A, B)$}\label{alg:max-sum-commmon-subsequence}
              \begin{algorithmic}[1]
                \Require{An array $A[1 \cdots n]$ of $n$ numbers and an array $B[1 \cdots m]$ of $m$ numbers.}
                \Ensure{The maximum common subsequence of $A$ and $B$.}
                \State $f[i][0] \gets 0$ \textbf{for each} $i \in \{0 \cdots n\}$
                \State $f[0][j] \gets 0$ \textbf{for each} $j \in \{0 \cdots m\}$
                \For{$i \gets 1$ \textbf{to} $n$}
                  \For{$j \gets 1$ \textbf{to} $m$}
                    \If{$A[i] = B[j]$}
                      \State $f[i][j] \gets f[i-1][j-1] + \max\big\{A[i], 0\big\}$
                    \Else
                    \State $f[i][j] \gets \max\big\{f[i][j-1], f[i-1][j]\big\}$
                    \EndIf
                  \EndFor
                \EndFor
                \State \Return $f[n][m]$
              \end{algorithmic}
            \end{algorithm}
          \end{minipage}
        \end{center}
    \end{enumerate}

  \newpage
  \item
    \begin{enumerate}
      \item
        {\itshape Algorithm Description} \\ \vspace{-4mm}

        We proceed with dynamic programming. Specifically, for any $1 \leq i \leq n$, we split the
        original problem into individual subproblem instances of the form $f[i]$ representing the
        length of the \textit{longest monotonically increasing subsequence} in $A[1 \cdots i]$
        that ends with $A[i]$. \\

        We compute the value of $f[i]$ by considering all of the $i-1$ elements in $A$ both previous
        to $A[i]$ and smaller than it, i.e., $\{j \in [1 \cdots i-1] \ni A[i] > A[j]\}$.
        These are the potential candidates that $A$ could be appended to in an increasing subsequence.
        Since we are after the longest one, we choose the $f[j]$ with maximal value, and set $f[i] =
        1 + f[j]$ (thereby effectively adding $A[i]$ to the end of the \textit{longest monotonically
        increasing subsequence} possible). So we have:
        \[
          f[i] = 1 + \max\big\{f[j] \ni 1 \leq j < i \land A[i] > A[j]\big\}
        \]
        The base case rests on $f[1] = 1$. Note $f[i]$ depends on all of the elements before it, so when we
        create our dynamic programming table, we start with the base case $f[1]$ and then progressively fill it from
        left to right. Once we determine the values for all $f[i]$, we just select the one with maximum
        value, and the algorithm is complete. Recovering the actual subsequence is a near trivial task
        of keeping track of which $A[i]$'s are added to the sequence and then backtrack them. The
        pseudocode for this is given below and has a time complexity of $O(n^2)$ incurred by the
        two \texttt{\color{red!50!black}for} loops in Steps $6$ and $8$.

      \item
        {\itshape Algorithm Pseudocode} \vspace{-4mm}
        \begin{center}
          \begin{minipage}{\linewidth}
            \begin{algorithm}[H]
              \caption{\textsc{Longest-Monotonically-Increasing-Subsequence}$(A)$}\label{alg:longest-monotonically-increasing-subsequence}
              \begin{algorithmic}[1]
                \Require{An array $A[1 \cdots n]$ of $n$ distinct numbers.}
                \Ensure{The longest monotonically increasing subsequence $S$ of $A$ and its length.}
                \State $l \gets 0$
                \State $k \gets 0$
                \State $S \gets \varnothing$
                \State $p[1 \cdots n] \gets 0$ \Comment{predecessors for each $i$'s -- useful for reconstructing $S$}
                \State
                \For{$i \gets 1$ \textbf{to} $n$}
                  \State $f[i] \gets 1$
                  \For{$j \gets 1$ \textbf{to} $i-1$}
                    \If{$A[i] > A[j]$ \textbf{and} $f[i] < f[j] + 1$}
                      \State $f[i] \gets f[j] + 1$
                      \State $p[i] \gets j$
                    \EndIf
                  \EndFor
                  \If{$l < f[i]$}
                    \State $l \gets f[i]$
                    \State $k \gets i$
                  \EndIf
                \EndFor
                \State
                \Repeat
                  \State $S \gets S \cup \{A[k]\}$
                  \State $k \gets p[k]$
                \Until{$k = 0$}
                \State
                \State $S \gets \textsc{Reverse}(S)$
                \State \Return $(S, l)$
              \end{algorithmic}
            \end{algorithm}
          \end{minipage}
        \end{center}
    \end{enumerate}
\end{enumerate}
\end{document}
