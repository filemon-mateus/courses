\documentclass[10pt, letterpaper]{article}
\usepackage{preamble}

\begin{document}
\header{CS $6150$}{Filemon Mateus}{Homework \#\ $3$}{Linear Programming}{\today}

\begin{enumerate}[label={\bfseries Q\arabic*.}]
  \item
    \begin{enumerate}
      \item
        Let $x^+$ denote the locations of the blocks with positive polarity, and let $x^-$ denote the location of the blocks with
        negative polarity. Also, let $y_{ij}$ represent a non-negative decision variable representing the length of the wire
        connecting terminal blocks $x^+_i$ and $x^-_j$. Using the $\ell_2$ norm, we define the linear program as follows:
        \begin{gather*}
          \text{minimize}\ \dsum_{i,j} y_{ij}\ \text{subject to} \\
          y_{ij} \geq \|x^+_i - x^-_j\|_2 \\
          y_{ij} \geq 0
        \end{gather*}
        i.e.
        \begin{gather*}
          \text{maximize}\ \dsum_{i,j} -y_{ij}\ \text{subject to} \\
          -y_{ij} \leq -\|x^+_i - x^-_j\|_2 \\
          y_{ij} \geq 0
        \end{gather*}
      \item
        Hence, its corresponding dual becomes:
        \begin{gather*}
          \text{minimize}\ \dsum_{i,j} -\|x^+_i - x^-_j\|_2\ z_{ij} \ \text{subject to} \\
          -z_{ij} \geq -1
        \end{gather*}
    \end{enumerate}

  \item
    Let $\bm{\mathcal{P}}$ denote the following linear program in $n$ variables: $\bm{\mathcal{P}} = \min\{c^Tx \mid Ax \geq b, x \in
    \mathbb{R}^n\}$. Then, by asymmetric duality, the \textit{primal} $\bm{\mathcal{P}}$ corresponds to the \textit{dual} $\bm{\mathcal{D}} =
    \max\{b^Ty \mid A^Ty = c, y \geq 0\}$ in $m$ variables. Suppose this \textit{dual} is feasible with maximum value $z$.
    Then, the following linear program $\bm{\mathcal{G}} = \{x \mid Ax \geq b, c^Tx \leq z\}$ is feasible only if $\bm{\mathcal{P}}$ is
    feasible with value at most $z$. This follows from the fact that $\min(c^Tx) \leq c^Tx \leq z$ and the existence
    of an optimum $\min(c^Tx) > z$ in $\bm{\mathcal{P}}$ violates the second linear constraint of $\bm{\mathcal{G}}$; thereby, making it
    infeasible. \\

    Now, if we partition the interval $[-M, +M]$ with a neighborhood around $z$ of the form $(z-\varepsilon, z+\varepsilon)$ with $\varepsilon > 0$, then the size
    of the search space for the optimal value of $\bm{\mathcal{P}}$ is precisely $M/\varepsilon$. This is because the interval $[-M, +M]$ has
    length/measure $2M$ and a neighborhood of the form $(z-\varepsilon, z+\varepsilon)$ has measure $2\varepsilon$. So
    partitioning $2M$ with $2\varepsilon$ produces precisely $M/\varepsilon$ neighborhoods---each of size $2\varepsilon$---where the optimal value of
    $\bm{\mathcal{P}}$ lands up to an error of $\pm\varepsilon$. \\

    Thus, the tentative solution here is to execute binary search on these $M/\varepsilon$ intervals, and check for feasibility on the extremities
    $z-\varepsilon$ and $z+\varepsilon$ using the \textbf{oracle}. In other words, guess $z = 0$ to be the optimal, then if $\bm{\mathcal{G}}$ is feasible
    at $z-\varepsilon$ search left; otherwise, if $\mathcal{G}$ is infeasible at $z+\varepsilon$ search right. If none of the aforementioned conditions
    evaluate, i.e. $\bm{\mathcal{G}}$ is infeasible at $z-\varepsilon$ and feasible at $z+\varepsilon$ (meaning that $z-\varepsilon < \min(c^Tx) \leq
    z+\varepsilon$) report $z\pm\varepsilon$ as the optimal value of $\bm{\mathcal{P}}$. \\

    Since we have an initial problem space of size $M/\varepsilon$, and each time we either search left or right, we're guaranteed to consult the
    \textbf{oracle} at most $O\left(\log\left(M/\varepsilon\right)\right)$ times.
\end{enumerate}
\end{document}
