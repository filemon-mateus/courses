\documentclass[10pt, letterpaper]{article}
% default base packages (we keep it minimal!)
\usepackage{algorithm}
\usepackage{algpseudocode}
\usepackage{amsmath}
\usepackage{amssymb}
\usepackage{amsthm}
\usepackage{bm}
\usepackage{cancel}
\usepackage{caption}
\usepackage{enumitem}
\usepackage{fancyhdr}
\usepackage{float}
\usepackage[margin=1in]{geometry}
\usepackage{graphicx}
\usepackage{hyperref}
\usepackage{lastpage}
\usepackage{mathpazo}
\usepackage{pgfplots}
\usepackage{subcaption}
\usepackage{thmtools}
\usepackage{tikz}
\usepackage{xcolor}

% useful stylistic settings for proofs, remarks, and claims
\let\proof\relax\let\endproof\relax
\declaretheoremstyle[
  headfont=\bfseries,
  bodyfont=\normalfont,
  spaceabove=\topsep,
  spacebelow=\topsep,
  postheadspace=.5em,
]{claim}
\declaretheoremstyle[
  headfont=\bfseries,
  bodyfont=\itshape,
  spaceabove=\topsep,
  spacebelow=\topsep,
  postheadspace=.5em,
]{remark}
\declaretheoremstyle[
  headfont=\itshape,
  bodyfont=\normalfont,
  spaceabove=\topsep,
  spacebelow=\topsep,
  postheadspace=.5em,
  qed=$\square$,
]{proof}

% enumerate proofs, remarks, and claims by default
\declaretheorem[style=proof,name=Proof,numbered=yes]{proof}
\declaretheorem[style=claim,name=Claim,numbered=yes]{claim}
\declaretheorem[style=remark,name=Remark,numbered=yes]{remark}

% unenumerate proofs, remarks, and claims on "*" envs
\declaretheorem[style=proof,name=Proof,numbered=no]{proof*}
\declaretheorem[style=claim,name=Claim,numbered=no]{claim*}
\declaretheorem[style=remark,name=Remark,numbered=no]{remark*}

% useful macros
\newcommand{\dsum}{\displaystyle\sum}
\algnewcommand\algorithmicto{\textbf{to}}
\algrenewtext{For}[3]{$\algorithmicfor\ #1 \gets #2\ \algorithmicto\ #3\ \algorithmicdo$}

% useful general settings
\captionsetup[figure]{font=normalsize,labelfont={bf},name={Fig.},labelsep=period}
\hypersetup{colorlinks,linkcolor={red!50!black},citecolor={green!50!black},urlcolor={blue!80!black}}

% personal custom header
\newcommand{\header}[5]{
  \pagestyle{fancy}
  \lhead{#2}
  \chead{#1~~---~~#4~~---~~#3}
  \rhead{#5}
  \cfoot{Page \textbf{\thepage} of \textbf{\pageref{LastPage}}}
}


\begin{document}
\header{Filemon Mateus}{Introduction \& Background}{Homework \#\ 0}

\begin{enumerate}
  \item[\textbf{Q1.}]
    \begin{remark*}
      Suppose that $f$ and $g$ are functions whose domain and co-domain are subsets
      of the real-numbers. Then $f(n)$ is $O(g(n))$ (read ``big-O of g'') if and only
      if there are positive real numbers $c$ and $n_0$ such that $0 \leq f(n) \leq 
      cg(n)$ for every $n \geq n_0$.
    \end{remark*}

    \begin{enumerate}
      \item
        \begin{claim*}
          \vspace{-6.7mm}
          $f(n) = n^2 + 5n + 20$ is $O(n^2)$.
        \end{claim*}

        \begin{proof*}
          Consider $c = 2$ and $n_0 = 25$. Then for any $n \geq n_0$, $n^2 \geq 25n
          \geq 5n + 20$. Since $n$ is positive, we also have $0 \leq n^2 + 5n + 20$.
          Therefore, for any $n \geq n_0$, $0 \leq n^2 + 5n + 20 \leq n^2 + n^2 = 2n^2
          = cn^2$. So, $n^2 + 5n + 20$ is $O(n^2)$.
        \end{proof*}

        \begin{claim*}
          $g(n) = 1/n^2 + 2/n$ is $O(1/n)$.
        \end{claim*}

        \begin{proof*}
          Consider $c = 3$ and $n_0 = 1$. Suppose that $n \geq n_0$. Then $n \geq 1$.
          So $0 \leq 1/n \leq 1$. Since $n$ is positive, this implies that $0 \leq 1/n^2
          \leq 1/n$. So then $0 \leq 1/n^2 + 2/n \leq 1/n + 2/n = 3/n = c \cdot 1/n$,
          which is what we needed to show.
        \end{proof*}

      \item
        Test \\
    \end{enumerate}

  \item[\textbf{Q2.}]
    Consider the following algorithm which yields the product of two $n-$digit numbers:
    \vspace{-6mm}
    \begin{center}
      \begin{minipage}{.99\linewidth}
        \begin{algorithm}[H]
          \caption{$\textsc{multiply}(a,b)$}\label{alg:multiply}
          \begin{algorithmic}
            \State $x \gets \textsc{Square}(a/2+b/2)$
            \State $y \gets \textsc{Square}(a/2-b/2)$
            \State $z \gets x - y$
            \State \Return{$z$}
          \end{algorithmic}
        \end{algorithm}
      \end{minipage}
    \end{center}

    \begin{claim*}
      Algorithm \textbf{\ref{alg:multiply}} is correct and produces the desired product $ab$
      in $O(n\log_{}n)$ time.
    \end{claim*}

    \begin{proof*}
      The proof is almost trivial, relying on the observation that Algorithm \textbf{\ref{alg:multiply}}
      always terminates with state $z$. Now state $z$ from the preceeding assignment is the
      difference $x - y$, which coincidently in turn is:
      \begin{align*}
        x -y 
          &= \bigg(\dfrac{a}{2} + \dfrac{b}{2}\bigg)^2 - \bigg(\dfrac{a}{2} - \dfrac{b}{2}\bigg)^2 \\
          &= \cancel{\bigg(\dfrac{a}{2}\bigg)^2} + 2\bigg(\dfrac{a}{2}\bigg)\bigg(\dfrac{b}{2}\bigg) + \cancel{\bigg(\dfrac{b}{2}\bigg)^2}
             -\cancel{\bigg(\dfrac{a}{2}\bigg)^2} + 2\bigg(\dfrac{a}{2}\bigg)\bigg(\dfrac{b}{2}\bigg) - \cancel{\bigg(\dfrac{b}{2}\bigg)^2} \\
          &= 4\bigg(\dfrac{a}{2}\bigg)\bigg(\dfrac{b}{2}\bigg) \\
          &= ab
      \end{align*}
      which is what we need to show! \\

      Now that we have established the algorithm correctness, we show its running time $T_M(n)$
      belongs to a collection of functions whose rate of growth are no larger than $n\log_{} n$.
      To do this, we observe that there are two calls to the \textsc{Square} subroutine (each of
      $n\log_{}n$ cost!), plus three assignments and a return statement which amount to some constant
      amount of work. So,

      \[
        T_M(n) = 2T_S(n) + O(1) = 2O(n\log_{}n) + O(1) = O(n\log_{}n).
      \]

      The last equality follows from the fact that $O(1)$ is also $O(n\log_{}n)$, so we get
      $3O(n\log_{}n) = O(n\log_{}n)$ as desired!
    \end{proof*}

  \item[\textbf{Q3.}]
  \item[\textbf{Q4.}]
    \begin{enumerate}
      \item 
        Let $\bm{X}$ be a random variable representing the total number of heads on $k$
        trials. Then $\bm{X} \sim \text{Bin}(k, 1/2)$. So, $\mathbb{P}(\bm{X} = 1) = 
        {}_kC_1(1/2)(1/2)^{k-1} = k/2^k$. \\

      \item
        $\mathbb{P}(\text{all $k$ boxes get distinct colors}) = \dfrac{{}_kC_1}{k} \cdot 
        \dfrac{{}_{k-1}C_1}{k} \cdot \dfrac{{}_{k-2}C_1}{k} \cdots \dfrac{{}_1C_1}{k} =
        \dfrac{k!}{k^k}$. \\

      \item
        Let $\bm{X}$ represent the length of the sequence of rolls until a $1$ is seen.
        Then, $\bm{X}$ follows a geometric distribution with pmf $\mathbb{P}(\bm{X} = k)
        = \left(1/6\right) \left(5/6\right)^{k-1}$. So,
        \[
          \mathbb{E}(\bm{X})
            = \dsum_{k=1}^{\infty} k\ \mathbb{P}(\bm{X} = k)
            = \dsum_{k=1}^{\infty} k \left(\dfrac{1}{6}\right) \left(\dfrac{5}{6}\right)^{k-1}
            = \dfrac{1}{6} \dsum_{k=1}^{\infty} k \left(\dfrac{5}{6}\right)^{k-1}
            = \dfrac{1}{6} \left(1-\dfrac{5}{6} \right)^{-2}
            = 6.
       \]
        The required number of rolls to observe a $1$ with probability greater than $99/100$
        is obtained by enforcing the inequality: 
        \[
          \dsum_{k=1}^{n}\mathbb{P}(\bm{X} = k) 
            = \dsum_{k=1}^{n} \left(\dfrac{1}{6}\right) \left(\dfrac{5}{6}\right)^{k-1}
            = \dfrac{1}{6} \dsum_{k=1}^{n} \left(\dfrac{5}{6}\right)^{k-1} > \dfrac{99}{100}.
        \]
        Simplyfying the latter expression, we get: 
        \[
          1 - \left(\dfrac{5}{6}\right)^{n} > \dfrac{99}{100} \implies n > \dfrac{\ln(100)}
          {\ln(6) - \ln(5)} \approx 25.26.
        \]
        So, at least $n = 26$ rolls are necessary!
    \end{enumerate}

  \bigskip

  \item[\textbf{Q5.}]
    \begin{enumerate}
      \item
        % \begin{remark*}
        %   If $\bm{X}$ is a r.v. with distribution $\text{Bin}(N, p)$, then provided $N$ is large
        %   enough, $\mathcal{N}(\mu, \sigma^2)$ is a good approximation for $\text{Bin}(N, p)$
        %   where $\mu = p$ and $\sigma^2 = np(1-p)$. \\
        % \end{remark*}

        From the problem statement it is evident that $\bm{H_1}$ and $\bm{H_2}$ are independent,
        identically distributed random variables with $\bm{H_1} \sim \text{Bin}(N, 1/2)$ and
        $\bm{H_2} \sim \text{Bin}(N, 51/100)$.  Assuming $N$ is sufficiently large, we model 
        $\bm{H_1}$ and $\bm{H_2}$ as normal distributions:
        \[
          \bm{H_1} \sim \mathcal{N}\left(\dfrac{N}{2}, \dfrac{N}{4}\right) \quad \text{and} \quad
          \bm{H_2} \sim \mathcal{N}\left(\dfrac{51}{100}N, \dfrac{51}{100}\dfrac{49}{100}N\right)
        \]

        Since $\bm{H_1}$ and $\bm{H_2}$ are independent normal, their difference $\bm{H_2} - \bm{H_1}$
        will also be normal:
        \[
          \bm{H_2} - \bm{H_1} \sim \mathcal{N}\left(\dfrac{N}{100}, \dfrac{N}{4} + \dfrac{51}{100}
          \dfrac{49}{100}N\right)
        \]
        % Recall, we are interested in a sufficiently large $N$ such that $\mathbb{P}(\bm{H_2} - \bm{H_1}
        % \geq 1) = 1$. Running a few experiments for varying values of $N$, we find $N = 67$ to be a
        % suitable choice.
        \begin{figure}[H]
          \centering
          \begin{tikzpicture}
            \definecolor{red}{HTML}{dc3522}
            \definecolor{gray}{HTML}{d4d4d4}
            \definecolor{green}{HTML}{008700}
            \begin{axis}[
              xlabel={$N$},
              ylabel={$\mathbb{P}(\bm{H_2}-\bm{H_1}\geq1)$},
              legend cell align={left},
              legend style={
                fill opacity=0.8,
                draw opacity=1,
                text opacity=1,
                at={(0.97,0.03)},
                anchor=south east,
              },
              tick align=outside,
              tick pos=left,
              x grid style={gray, dotted},
              xmajorgrids,
              xtick style={color=black},
              y grid style={gray, dotted},
              ymajorgrids,
              ytick style={color=black},
            ]
              % \addplot[very thick, color=red] table[col sep=comma]{data/hw0-sim.csv};
              \addlegendentry{$\mathbb{P}(\bm{H_2}-\bm{H_1}\geq1)$}
            \end{axis}
          \end{tikzpicture}
        \end{figure}

      \item
        By linearity of expectations, $\mathbb{E}(\bm{H_2} - \bm{H_1}) = \mathbb{E}(\bm{H_2})
        - \mathbb{E}(\bm{H_1}) = 25\left(\dfrac{51}{100} \right) - 25\left(\dfrac{1}{2}\right)
        = \dfrac{1}{4}.$ \\

      \item 
        By Markov Inequality, $\mathbb{P}(\bm{H_2} - \bm{H_1} \geq 1) \leq \mathbb{E}(\bm{H_2}
        - \bm{H_1})$
    \end{enumerate}

  \item[\textbf{Q6.}]
    Consider the following algorithm which returns (the indices of) a triplet with zero sum in
    a collection of intengers $A[1 \ldots n]$.
    \vspace{-6mm}
    \begin{center}
      \begin{minipage}{.99\linewidth}
        \begin{algorithm}[H]
          \caption{$\textsc{find-triplet}(A)$}\label{alg:find-triplet}
          \begin{algorithmic}
            \State $n \gets$ \Call{\bfseries size}{$A$}
            \If{$n < 3$}
              \State \Return{none}
            \EndIf
            \State \Call{\bfseries sort}{$A$}
            \For{i}{1}{n-2}
              \State $j \gets i + 1$
              \State $k \gets n$
              \While{$j < k$}
                \If{$A[i] + A[j] + A[k] < 0$}
                  \State $j \gets j + 1$
                \ElsIf{$A[i] + A[j] + A[k] > 0$}
                  \State $k \gets k - 1$
                \Else
                  \State \Return{$i,j,k$}
                \EndIf
              \EndWhile
            \EndFor
            \State \Return{none}
          \end{algorithmic}
        \end{algorithm}
      \end{minipage}
    \end{center}

    \begin{claim*}
      $T(n)$ for the ``\textsc{find-triplet}'' algorithm is $O(n^2)$.
    \end{claim*}

    \begin{proof*}
      The proof is almost trivial, relying on the observation that Algorithm \textbf{\ref{alg:find-triplet}}
      always terminates with state $z$. Now state $z$ from the preceeding assignment is the
      difference $x - y$, which coincidently in turn is:
    \end{proof*}
\end{enumerate}
\end{document}
